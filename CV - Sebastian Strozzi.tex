%%%%%%%%%%%%%%%%%%%%%%%%%%%%%%%%%%%%%%%%%%%%%%%%%%%%%%%%%
% CV di Sebastian Strozzi - File Principale
% Questo file è solo per il contenuto.
% Tutta la formattazione è gestita da "impostazioni.tex"
%%%%%%%%%%%%%%%%%%%%%%%%%%%%%%%%%%%%%%%%%%%%%%%%%%%%%%%%%

\documentclass[10pt, a4paper]{article}

% --- RICHIAMA TUTTE LE IMPOSTAZIONI, PACCHETTI E STILI ---
%%%%%%%%%%%%%%%%%%%%%%%%%%%%%%%%%%%%%%%%%%%%%%%%%%%%%%%%%
% File di Impostazioni per CV
% Questo file contiene tutta la formattazione e lo stile.
%%%%%%%%%%%%%%%%%%%%%%%%%%%%%%%%%%%%%%%%%%%%%%%%%%%%%%%%%



%%% --- PACCHETTI FONDAMENTALI ---
\usepackage[utf8]{inputenc}     % Codifica del testo per accenti, etc.
\usepackage[italian]{babel}     % Sillabazione e nomi in italiano
\usepackage[T1]{fontenc}        % Migliore gestione dei font e dei caratteri speciali


%%% --- GESTIONE FONT ---
\usepackage{lmodern} % Font di default, pulito ma dà problemi (es. grassetto+maiuscoletto)
% \usepackage{fourier}   % CONSIGLIATO: Font più completo e professionale, risolve problemi di stile


%%% --- PACCHETTI PER GRAFICA E COLORI ---
%\usepackage{microtype}			% Aggiunto per non andare a capo???????
\usepackage{graphicx}           % Per inserire immagini (la tua foto)
\usepackage[dvipsnames]{xcolor} % Per gestire colori avanzati (es. HTML, nomi di colori)

%%% --- GEOMETRIA DELLA PAGINA E LAYOUT ---
\usepackage[
a4paper,
left=1.8cm,
right=1.8cm,
top=1.8cm,
bottom=1.3cm,
nohead,
nofoot
]{geometry}


%%% --- PACCHETTI PER STILE AVANZATO ---
\usepackage{titlesec}           % Per personalizzare i titoli delle sezioni
\usepackage{hyperref}           % Per rendere i link (email, web) cliccabili
\usepackage[shortlabels]{enumitem} % Per personalizzare gli elenchi puntati




%%% --- DEFINIZIONE DEGLI STILI PERSONALIZZATI ---


% -- COLORE DI LINK E TITOLI SEZIONI --
% Definisci qui i colori principali per cambiarli facilmente in tutto il documento
\definecolor{headingcolor}{HTML}{003B4C}
\definecolor{linkcolor}{HTML}{003B4C} % Un blu scuro professionale


% -- CONFIGURAZIONE LINK --
\hypersetup{
	colorlinks=true,
	urlcolor=linkcolor,
	linkcolor=linkcolor,
	}


% -- SPAZIATURA --
\setlength{\parskip}{0em}
\setlength{\parindent}{0em}


% -- TITOLI DI SEZIONE --
% Definisci lo spessore della linea sotto i titoli.
\newcommand{\barraTitoli}{{\titlerule[0.2pt]}}

% Definisci il formato per le sezioni non numerate (\section*)
\titleformat
{name=\section,numberless}
{\Large\scshape\raggedright\color{headingcolor}} % Formato del testo del titolo
{} % Spazio per il numero (vuoto)
{0em} % Spazio tra numero e titolo
{} % Codice prima del titolo (vuoto)
[\barraTitoli\vspace{7pt}] % Codice dopo il titolo (linea + spazio verticale)
\titlespacing{\section}{0pt}{2ex}{1ex}

% -- ELENCHI PUNTATI --

% Definisci lo stile per renderli più compatti e con un pallino
\setlist[itemize]{
	leftmargin=*,
	label=\textbullet,
	itemsep=0.2em,
	topsep=0.5em
	}


%%% --- FINE DELLE IMPOSTAZIONI ---




\begin{document}
		\thispagestyle{empty}
	% --- INTESTAZIONE PRINCIPALE ---
	
	\begin{center}
		% --- NOME ---
		{\fontsize{30pt}{36pt}\selectfont \scshape Sebastian Strozzi} \\ \vspace{3mm}
		% --- INFO BASE ---	
		Capriate S.Gervasio (BG), Italia \textbullet\ (+39) 333 2869374 \textbullet\  \href{mailto:strozzisebastian@gmail.com}{strozzisebastian@gmail.com}  \\ \vspace{2pt}
		\href{https://github.com/sebastianstrozzi}{github.com/sebastianstrozzi} \textbullet\ \href{https://www.linkedin.com/in/sebastian-strozzi}{linkedin.com/in/sebastian-strozzi}		   		
	\end{center}
	
	\vspace{2mm}
	
	
	% --- LAYOUT A DUE COLONNE ---
	
	\begin{minipage}[t]{0.35\textwidth}	          % --- COLONNA SINISTRA ---

		% --- FOTOTESSERA ---
		\phantom{ciao!}		%<------ questo carattere fantasma risolve l'allineamento della fototessera!
		\begin{center}
			\includegraphics[width=0.7\linewidth]{foto.jpg} % L'immagine deve essere nella stessa cartella di questo file (per il futuro: sposta in una soot-cartela assieme al file di impostazioni)
		\end{center}
		\vspace{-3pt}
			
		\section*{Competenze}
			
			\textbf{Linguaggi e Software} \vspace{3pt} \\ 
			Linguaggio LaTeX e Fogli di Calcolo, Programmazione a Oggetti con Java e Calcolo Scientifico con MATLAB.
			\vspace{3mm} \\
			%
			\textbf{Aree di Competenza} \\
			Logica formale, problem-solving con progettazione di algoritmi, creazione di modelli matematici, analisi statistica e rielaborazione dati.
			\vspace{3pt}
			
		\section*{Istruzione}
		
			\textbf{Laurea Triennale in Matematica} \\
			\textit{Università degli Studi di Milano-Bicocca} \vspace{0pt} \\
			Tesi in Logica Matematica sui sistemi formali e la loro coerenza.
			\vspace{3pt}
			
		\section*{Lingue}
		
			\textbf{Italiano:} Madrelingua \\
			\textbf{Inglese:} Livello C1
			\vspace{3pt}
			
		\section*{Interessi}
		
			Assemblaggio di PC e ottimizzazione di sistemi come l'overclocking da BIOS. \\ Videogiochi (con modifiche all'engine.ini per aumentarne qualità e prestazioni).\\ Piccoli sistemi audio. \\
			Sport come alpinismo e arrampicata che richiedono concentrazione, capacità di pianificazione e gestione del rischio.
			
	\end{minipage}
	%
	\hspace{0.03\textwidth}								% Spazio tra le colonne
	%
	%
	\begin{minipage}[t]{0.62\textwidth}					% --- COLONNA DESTRA ---
		
		\phantom{\small ciao!}	\vspace{-6pt}			% <------ qui per ribilanciare partenza testo		
		\section*{Profilo Professionale}
		
			Sono un laureato in Matematica con esperienza nella progettazione di sistemi logici, dalla creazione di documenti complessi in LaTeX alla modellazione e analisi dati su Calc. Sto focalizzando il mio percorso sull'acquisizione di competenze avanzate e l'apprendimento di nuovi strumenti e linguaggi. Il mio obiettivo è di applicare con efficacia le mie capacità a progetti futuri e problemi complessi, come il campo dell'Intelligenza Artificiale.
		
		
		\section*{Progetti e Articoli}
		
		\textbf{Sviluppo di Strumenti di Analisi e Automazione} \\
				\textit{Progetti professionali su fogli di calcolo}
		\begin{itemize}
			\item Strumento per l'analisi dei voti scolastici con statistiche, individualizzazione, controlli di errori e formattazione avanzata.
			\item Strumento di aiuto nella creazione manuale dell'orario di un piccolo istituto: implementazione di controlli incrociati, ottimizzazioni e formattazione avanzata per migliorarne l'efficacia.
		\end{itemize}
		\vspace{2mm}
		
		\textbf{Prototipo per ottimizzazione di Sistemi a Regole} \\
		\textit{Progetto personale di modellazione algoritmica}
		\begin{itemize}
			\item Progettazione di un modello algoritmico per un sistema a regole complesse con variabili interdipendenti in ambiente videoludico.
		\end{itemize}
		\vspace{2mm}
			
		\textbf{La Consistenza dell'Aritmetica di Peano} \\
		\textit{Tesi di Laurea Triennale}
		\begin{itemize}
			\item Analisi dei fondamenti della matematica e rielaborazione personale critica di una delle dimostrazioni fondamentali di G. Gentzen. 
			\item Vengono esplorati anche concetti alla base della teoria dei linguaggi informatici e della deduzione automatica.
		\end{itemize}
		\vspace{2mm}
		
		\textbf{Dispense Didattiche} \\
		\textit{Selezione di articoli di approfondimento scritti per gli studenti}
		\begin{itemize}
			\item Sviluppo di materiale didattico per l'esposizione chiara ed accessibile di approfondimenti di livello universitario (tra cui Algebra Lineare, Teoria dei Grafi e Modelli Probabilistici) e la loro applicazione a contesti reali, tra cui la biologia molecolare e la filogenetica.
		\end{itemize}
		
		
			
		\section*{Esperienza Lavorativa}
		
			\textbf{Docente di Matematica e Fisica} \hfill [2021 - 2025] \\
			\textit{Ministero dell'Istruzione, Bergamo}
			\begin{itemize}
				\item Progettazione e gestione di percorsi didattici per classi fino a 30 studenti, adattando i metodi di insegnamento alle performance.
				\item Sviluppo di fogli di calcolo per migliorare la fase valutativa, calibrare le richieste e ottenere un riscontro sull'efficacia dei metodi adottati.
				\item Traduzione di concetti complessi in strutture logiche e intuitive.
			\end{itemize}
		
	\end{minipage}

\end{document}