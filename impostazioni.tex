%%%%%%%%%%%%%%%%%%%%%%%%%%%%%%%%%%%%%%%%%%%%%%%%%%%%%%%%%
% File di Impostazioni per CV
% Questo file contiene tutta la formattazione e lo stile.
%%%%%%%%%%%%%%%%%%%%%%%%%%%%%%%%%%%%%%%%%%%%%%%%%%%%%%%%%



%%% --- PACCHETTI FONDAMENTALI ---
\usepackage[utf8]{inputenc}     % Codifica del testo per accenti, etc.
\usepackage[italian]{babel}     % Sillabazione e nomi in italiano
\usepackage[T1]{fontenc}        % Migliore gestione dei font e dei caratteri speciali


%%% --- GESTIONE FONT ---
\usepackage{lmodern} % Font di default, pulito ma dà problemi (es. grassetto+maiuscoletto)
% \usepackage{fourier}   % CONSIGLIATO: Font più completo e professionale, risolve problemi di stile


%%% --- PACCHETTI PER GRAFICA E COLORI ---
%\usepackage{microtype}			% Aggiunto per non andare a capo???????
\usepackage{graphicx}           % Per inserire immagini (la tua foto)
\usepackage[dvipsnames]{xcolor} % Per gestire colori avanzati (es. HTML, nomi di colori)

%%% --- GEOMETRIA DELLA PAGINA E LAYOUT ---
\usepackage[
a4paper,
left=1.8cm,
right=1.8cm,
top=1.8cm,
bottom=1.3cm,
nohead,
nofoot
]{geometry}


%%% --- PACCHETTI PER STILE AVANZATO ---
\usepackage{titlesec}           % Per personalizzare i titoli delle sezioni
\usepackage{hyperref}           % Per rendere i link (email, web) cliccabili
\usepackage[shortlabels]{enumitem} % Per personalizzare gli elenchi puntati




%%% --- DEFINIZIONE DEGLI STILI PERSONALIZZATI ---


% -- COLORE DI LINK E TITOLI SEZIONI --
% Definisci qui i colori principali per cambiarli facilmente in tutto il documento
\definecolor{headingcolor}{HTML}{003B4C}
\definecolor{linkcolor}{HTML}{003B4C} % Un blu scuro professionale


% -- CONFIGURAZIONE LINK --
\hypersetup{
	colorlinks=true,
	urlcolor=linkcolor,
	linkcolor=linkcolor,
	}


% -- SPAZIATURA --
\setlength{\parskip}{0em}
\setlength{\parindent}{0em}


% -- TITOLI DI SEZIONE --
% Definisci lo spessore della linea sotto i titoli.
\newcommand{\barraTitoli}{{\titlerule[0.2pt]}}

% Definisci il formato per le sezioni non numerate (\section*)
\titleformat
{name=\section,numberless}
{\Large\scshape\raggedright\color{headingcolor}} % Formato del testo del titolo
{} % Spazio per il numero (vuoto)
{0em} % Spazio tra numero e titolo
{} % Codice prima del titolo (vuoto)
[\barraTitoli\vspace{7pt}] % Codice dopo il titolo (linea + spazio verticale)
\titlespacing{\section}{0pt}{2ex}{1ex}

% -- ELENCHI PUNTATI --

% Definisci lo stile per renderli più compatti e con un pallino
\setlist[itemize]{
	leftmargin=*,
	label=\textbullet,
	itemsep=0.2em,
	topsep=0.5em
	}


%%% --- FINE DELLE IMPOSTAZIONI ---

